\documentclass {article}
\usepackage{fullpage}

\begin{document}

~\vfill
\begin{center}
\Large

A5 Project Proposal

Title: Ellipsoidal coordinates in Computer Graphics

Name: Adam French

Student ID:21111227

UserId: a3french
\end{center}
\vfill ~\vfill~
\newpage
\section{Purpose}
     I wish to experiment with different coordinate systems for representing computer graphic scenes.
     The rationale behind this is mostly artistic, however depending on the coordenate system, I believe it could also have scientific implicatons.
     I would also like to pair this concept with fractal geometry for more intresting representations of minimal surfaces.
     Using taylor aproximations for minimal surfaces, I propose that I will also be researching the computational complexity of aproximating surfaces.
     

\section{Statement}
     My final product will be a scene rendering a number of minimal surfaces.

	The project is centered around the mathematics behind rendering primitives.
	We have seen efficient methods of computing intersections for spheres, cylinders and cubes.
	I wish to implement a larger range of mathematicially based primitives for rendering.

	I will program in a manner that

	This is an interesting challange as I have taken inspiration from the website "Virtual Math Mueseum".
	It has a collection of very interesting minimal surfaces, and I believe that I would be able to recreate these with phong and traditional raytracing procedures.

	I am hoping to understand more about changing coordinates systems and how it affects these mathematical surfaces.
     I am also hoping to understand the increase in computational complexity from aproximating these surfaces.

\section{Technical Outline}
    Basically, your objectives in your objective list should be fairly
    short statements of the objective; you should provide additional
    details about your objectives in this section to clarify what you
    plan to do.

     Further, survey the important data structures and algorithms that
     will be necessary to achieve the goals, and (for ray tracing
     projects) lists the new command
     that will need to be added to the input language.

     To  get  bold face: {\bf bold face words}.  To get italics: {\it italic
     face words}.  To  get typewriter font: {\tt typed words}.  To get
     larger  words:  {\large large  words}.   To  get smaller words: 
     {\small small words}.  

\section{Bibliography}
     Ellipsoidal coordenates (wolfram): https://mathworld.wolfram.com/ConfocalEllipsoidalCoordinates.html
     Ellipsoidal coordinates (3D): https://en.wikipedia.org/wiki/Ellipsoidal\_coordinates
     Elliptic coordinates (2D): https://en.wikipedia.org/wiki/Elliptic\_coordinate\_system
     Hyperbolica Game: https://store.steampowered.com/app/1256230/Hyperbolica/
     https://www.philipzucker.com/ray-tracing-algebraic-surfaces/
     Amazing maths website: https://virtualmathmuseum.org/

\newpage


\noindent{\Large\bf Objectives:}

{\hfill{\bf Full UserID:a3french\rule{2in}{.1mm}}\hfill{\bf Student ID:2111227\rule{2in}{.1mm}}\hfill}

\begin{enumerate}
     \item[\_\_\_ 1:]  Binary Space Partition
          Implement binary space partitioning to increase the efficiency of rendering multiple primitives.

     \item[\_\_\_ 2:]  Creating Scherk Surface intersection and normal.
          


     \item[\_\_\_ 3:]  Catenoid intersection and normal.

     \item[\_\_\_ 4:]  Psuedo-Sphere & Conic surface?

     \item[\_\_\_ 5:]  Fractals

     \item[\_\_\_ 6:]  Multi Threading

     \item[\_\_\_ 7:]  Anti-aliasing

     \item[\_\_\_ 8:]  Set up an interesting scene

     \item[\_\_\_ 9:]  Preview rendering (Only render a subset of rays)

     \item[\_\_\_ 10:]  Objective ten.
\end{enumerate}

% Delete % at start of next line if this is a ray tracing project
% A4 extra objective:
\end{document}
